%%%%%%%%%%%%%%%%%%%%%%%%%%%%%%%%%%%%%%%%%%%%%%%%%%%%%%%%%%
%
% Vzor pro sazbu kvalifikační práce
%
% Západočeská univerzita v Plzni
% Fakulta aplikovaných věd
% Katedra informatiky a výpočetní techniky
%
% Petr Lobaz, lobaz@kiv.zcu.cz, 2016/03/14
%
%%%%%%%%%%%%%%%%%%%%%%%%%%%%%%%%%%%%%%%%%%%%%%%%%%%%%%%%%%

% Možné jazyky práce: czech, english
% Možné typy práce: BP (bakalářská), DP (diplomová)
\documentclass[english,BP]{thesiskiv}

% Definujte údaje pro vstupní strany
%
% Jméno a příjmení; kvůli textu prohlášení určete, 
% zda jde o mužské, nebo ženské jméno.
\author{Roman Kalivoda}
\declarationmale

%alternativa: 
%\declarationfemale

% Název práce
\title{Extension of neural network architecture}

% 
% Texty abstraktů (anglicky, česky)
%
\abstracttexten{Spiking neural networks (SNNs) are artificial neural networks designed to mimic sparse and asynchronous nature of information processing observed in biology. In recent years, deep spiking networks emerged with efforts to draw on experiences with classic deep networks. There also appeared attempts to reuse the available state-of-the-art analogue neural networks (ANNs) completely, and replace the neurons for inference. This thesis contributes with analysis of the dominant methods used in the development of SNNs, and comparison of the major SNN simulation platforms. An application of the SNNs was demonstrated on detection of event-related potentials in EEG data.}

\abstracttextcz{Impulsní neuronové sítě jsou variantou umělých neuronových sítí, které jsou navrženy, aby simulovaly přirozenou rozptýlenost a asynchronii pozorovanou u biologických neuronových sítí. Pokrok v nedávné době umožnil vytváření vícevrstvých impulsních sítí. S tím se objevila i snaha dosáhnout podobných úspěchů jako s klasickými vícevrstvými sítěmi. Objevily se také snahy o opětovné využití již existujících a populárních architektur klasických neuronových sítí. Ty mohou být například použity během procesu učení a nahrazeny impulsními sítěmi až v provozu. Tato práce analyzuje hlavní metody využité při vytváření impulsních sítí a porovnává nástroje pro jejich simulaci. Také bylo ukázáno použití impulsní neuronové sítě na detekci kognitivních evokovaných potenciálů v EEG datech.}

% Na titulní stranu a do textu prohlášení se automaticky vkládá 
% aktuální rok, resp. datum. Můžete je změnit:
%\titlepageyear{2016}
%\declarationdate{1. března 2016}

% Ve zvláštních případech je možné ovlivnit i ostatní texty:
%
%\university{Západočeská univerzita v Plzni}
%\faculty{Fakulta aplikovaných věd}
%\department{Katedra informatiky a výpočetní techniky}
%\subject{Bakalářská práce}
%\titlepagetown{Plzeň}
%\declarationtown{Plzni}

%%%%%%%%%%%%%%%%%%%%%%%%%%%%%%%%%%%%%%%%%%%%%%%%%%%%%%%%%%
%
% DODATEČNÉ BALÍČKY PRO SAZBU
% Jejich užívání či neužívání záleží na libovůli autora 
% práce
%
%%%%%%%%%%%%%%%%%%%%%%%%%%%%%%%%%%%%%%%%%%%%%%%%%%%%%%%%%%
% for inserting landscape oriented pages
\usepackage{pdflscape}

% Enables long table to be split
%\usepackage{longtable}

% More adjustable table environment
\usepackage{tabularx}

% Professional looking tables
\usepackage{booktabs}

% Improves justification
\usepackage{microtype}

% balík pro import dílčích .tex souborů do hlavního
%\usepackage{import}

\usepackage{csquotes}

% Zařadit literaturu do obsahu
\usepackage[nottoc,notlot,notlof]{tocbibind}

% Umožňuje vkládání obrázků
\usepackage[pdftex]{graphicx}

% SVG image support
\usepackage{svg}

% To rotate the appendices
\usepackage{rotating}

\graphicspath{ {./images/} }

% Odkazy v PDF jsou aktivní; navíc se automaticky vkládá
% balíček 'url', který umožňuje např. dělení slov
% uvnitř URL
\usepackage[pdftex]{hyperref}
\hypersetup{colorlinks=true,
  unicode=true,
  linkcolor=black,
  citecolor=black,
  urlcolor=black,
  bookmarksopen=true}
\usepackage{cleveref}

\usepackage{listings}
\crefname{lstlisting}{listing}{listings}
\Crefname{lstlisting}{Listing}{Listings}

% Při používání citačního stylu csplainnatkiv
% (odvozen z csplainnat, http://repo.or.cz/w/csplainnat.git)
% lze snadno modifikovat vzhled citací v textu
\usepackage[backend=biber, style=ACM-Reference-Format]{biblatex}

\addbibresource{references.bib}
%\usepackage[numbers,sort&compress]{natbib}

%%%%%%%%%%%%%%%%%%%%%%%%%%%%%%%%%%%%%%%%%%%%%%%%%%%%%%%%%%
%
% VLASTNÍ TEXT PRÁCE
%
%%%%%%%%%%%%%%%%%%%%%%%%%%%%%%%%%%%%%%%%%%%%%%%%%%%%%%%%%%
\begin{document}
%
\maketitle
\tableofcontents
% V souboru \texttt{literatura.bib} jsou uvedeny příklady, jak citovat knihu \cite{KnuthAOCP2}, článek v časopisu \cite{Hoare1961}, webovou stránku \cite{Graphics2D}.

%%%%%%%%%%%%%%%%%%%%%%%%%%%%%%%%%%%%%%%%%%%%%%%%%%%%%%%%%%
% CHAPTERS
%%%%%%%%%%%%%%%%%%%%%%%%%%%%%%%%%%%%%%%%%%%%%%%%%%%%%%%%%%

\chapter{Introduction} \label{chap:01}

A neural network is a set or population of specialised cells (neurons) that are interconnected by synapses. In biology, the neural network forms the structure of a neural system in animals. The interconnection pattern, size and spatial organisation of the population determines the architecture and specialisation of the network. The network specialises in carrying out a specific function when it is activated. These specialised networks then tie to one another to develop larger systems (e.g. animal brain). These natural phenomena inspired computer scientists to design a set of algorithms that models some aspects of the animal brain. 

Artificial (analogue) neural networks (ANNs) were initially inspired by biological neural systems, but many concepts were simplified or modified to conform to their practical applications. On the other hand, the spiking neural networks (SNNs) were mostly meant to simulate their biological models as closely as possible and thus improve our understanding of how real biological networks achieve their cognitive abilities with incredibly low energy consumption. Although the differences between those approaches still last, some aspects of the two approaches are getting more interconnected as the research advances and one approach can exploit the other. Nowadays, deep ANNs achieve satisfying accuracy in many complex tasks, although with extensive requirements on computation power. Because of that, new goals for the improvement of their efficiency are being set. That is where deep SNNs appear to be an exciting topic of research because of better power efficiency \cite{caoSpikingDeepConvolutional2015, tavanaeiDeepLearningSpiking2019}, although the applications generally do not accomplish the same accuracy yet. Apart from the applications in which ANNs are used, spiking networks open up new fields of utilisation. In \cref{chap:02} is a summary of the principal differences between the two classes of neural networks and a survey of the current state of the research in the area of the spiking neural networks. \Cref{chap:03} compares existing simulators of the spiking neural networks. In \cref{chap:04} are described the remaining tools used in this work. \Cref{chap:05} describes early attempts to simulate spiking networks with selected software.

\chapter{Basic description of the analogue neural networks and spiking neural networks} \label{chap:02}

\section{Artificial neural networks}
Artificial neural networks are built of idealised units which compute output values from their weighted input values and continuous nonlinear activating functions \cite{tavanaeiDeepLearningSpiking2019}. The units interact with each other by propagating its continuous output. These units connect themselves into distinct layers. If there is more than one hidden layer (hidden layer is the layer which takes input from the previous layer and outputs own results to another layer, that means it is neither input layer, nor output layer), the network is recognised as a deep network. In deep networks, each layer works with a set of features received from the output of the previous layer. This feature hierarchy enables deep networks to operate accurately on large data sets with plenty of parameters. Advances in the hardware computation power made it possible to use very deep networks for various artificial intelligence tasks. However, the (still quite) significant computation costs of these networks prevent them from applications in embedded and power-constrained environments. \par
The resulting neural network needs to be learned to become functional. Learning neural network means adjusting its connection weights and other parameters to optimise the accuracy of the results. There exist two major approaches to learning of a neural network, supervised learning and unsupervised learning.

\subsection{Supervised learning in artificial neural networks}
A prerequisite for the supervised learning method is a labelled data set. Supervised learning is useful for classification or regression tasks when we need to find a set of significant relationships or structure in the input data, that means we need to approximate a regression function. The functionality of the method is dependent on the complexity of the approximated function. Higher complexity is needed to learn the full structure of the data, but there is a threat of over-fitting the model, that means the model is too complicated and fails on general data. In the neural network, the supervised learning method takes advantage of the error backpropagation algorithm. The algorithm learns the network on the learning data set and finds the difference between the computed and desired output. Then the synaptic weights are adjusted to minimise the difference \cite{sathyaComparisonSupervisedUnsupervised2013}.

\subsection{Unsupervised learning in artificial neural networks}
Unsupervised learning means that the data set is missing labels, and the model attempts to find some structure in the underlying data set. Common use-cases for unsupervised learning are exploratory analysis and dimensional reduction. Exploratory analysis can get a fundamental inspection of the data in situations where it is difficult or impossible for people to find trends in the data. Typical applications for unsupervised learning methods are data clustering or visualisation. Another example of unsupervised learning is generative adversarial networks (GANs). GANs is an architecture that uses two neural networks to generate new synthetic data with a structure similar to real data \cite{goodfellowGenerativeAdversarialNets2014}.

\section{Spiking neural networks}
Although the biological neural networks inspired the basic principles of artificial neural networks, the processes in the biological networks differ from the processes in ANNs. The spiking neural networks proceed from the knowledge retrieved from observations of the biological neural networks. The basic idea behind SNN is that the network consists of the spiking neurons which have a different characteristic than artificial neurons. Each spiking neuron has a specified threshold value and membrane potential. Note that the resting membrane potential of the real biological neuron is typically negative. However, for mathematical simplification, it is convenient to assume in the spiking neuron models, that the resting membrane potential is zero, and thus the current membrane potential is the sum of postsynaptic potentials \cite{maassNetworksSpikingNeurons1997}. The postsynaptic potential is a result of an action potential (spike) generated by previous neurons which connects to the observed neuron. The postsynaptic potential can be either excitatory or inhibitory. The excitatory postsynaptic potential (EPSP) increments (depolarise) the membrane potential and the inhibitory postsynaptic potential decrements (hyper-polarise) the membrane potential. When the membrane potential reaches the threshold value of the neuron, a spike (a sudden increase of voltage) is generated. It can also be said that the neuron fired. Depending on the level of detail, various models of spiking neuron exist. Popular spiking neuron models are the leaky integrate-and-fire model (LIF), the Izhikevich neuron model or spike response model \cite{tavanaeiDeepLearningSpiking2019}. \par
Besides the biological perspective on spiking neural networks, there also exists a more technical point of view used in neuromorphic engineering. In this perspective, the spikes are more often called events. This term originates from the address event representation protocol \cite{pazTestInfrastructureAddressEventRepresentation2005, boahenPointtopointConnectivity00} which is used to connect event-based neuromorphic peripherals. With this point of view, the emphasis on the biological plausibility of the model and level of detail of the neurons steps back to allow more pragmatic ways to use the networks in such neuromorphic applications. The combination of the spiking neural networks with event-based sensors is especially useful because it enables both parts to utilise its full potential in terms of power efficiency \cite{pfeifferDeepLearningSpiking2018}.

\section{Deep spiking neural networks}
Similar to ANNs, if the spiking neural network consists of multiple layers, and at least one of them is hidden, the network is called deep. Deep neural networks, both spiking and non-spiking, are more capable. The interconnected neurons transfer information between themselves by firing trains of spikes (spike trains). Strictly speaking, the information is coded in the number of spikes and their frequencies, rather than amplitude characteristic of a single spike. That means the communication in the network is discrete in time in contrast to ANN, where the communication is going on continuous activation functions. This characteristic makes the spiking neural networks interesting to research because SNNs can be used to create more efficient low-energy consuming neural networks. Also, SNNs can be implemented on specialised dedicated hardware, which even more improves energy efficiency. Another advantage of the deep spiking networks is that the approximation of the final layer output is retrievable at the time of recording the first input spikes and the approximation improves over time \cite{pfeifferDeepLearningSpiking2018}.
Deep spiking neural networks have theoretically the same representational power as deep ANNs with lower energy requirements. Nevertheless, there is a problem with achieving the same results as with ANNs because optimal solutions for supervised learning of the SNNs do not exist yet. The gradient-based optimisation methods used in the artificial neural networks require the activation function to be differentiable \cite{tavanaeiDeepLearningSpiking2019}. The spike trains generated in spiking neurons can be formally represented by sums of delta functions which do not have derivatives. Multiple approaches proposed how to overcome this problem, but the research of the learning methods for the SNNs is still at its beginning \cite{tavanaeiDeepLearningSpiking2019}. 

\section{State of the art}
Recently there appeared many novel theories and experiments on the topic of the deep spiking neural networks and new approaches to learning such networks. As stated above, traditional methods for deep learning can be hardly used because of the distinct characteristics of the two models. One of the approaches to overcome this issue is using some sort of approximate derivatives or other substitutes, although this may reduce bio-plausibility of the model. \cite{leeTrainingDeepSpiking2016} proposed a spiking network backpropagation rule with low-pass filtering to handle discontinuity at the time of the spike. This method demonstrates state-of-the-art results for deep SNNs.\par
The more biologically plausible approach is using local learning rules such as spike-timing-dependent plasticity (STDP). STDP adjusts the synaptic weights between presynaptic and postsynaptic neuron according to the relative difference of their spike times. If presynaptic neuron fires relatively briefly before the postsynaptic neuron, their synaptic weight increments. The weight decrements in the opposite case, that is in the case when presynaptic neuron spikes briefly after the postsynaptic neuron. Local learning rules can be used with unsupervised learning \cite{tavanaeiDeepLearningSpiking2019}, but with supervised learning, there is once again missing a plausible backpropagation method. This problem can be covered, for example, by introducing recurrent connections to propagate the error signal. The use of local learning rules is also interesting for efficient implementations on dedicated hardware, such as SpiNNaker\cite{furberSpiNNakerProject14} or BrainScales. At present, the efficiency of the networks which use this method falls behind in terms of precision. \par
Another approach is to create a conversion between ANN and SNN. For the conversion, a conventional deep network trains using backpropagation and other methods available in ANNs. Then the already trained network is transformed into its spiking variant. The conversion is done by mapping the analogue neurons onto spiking ones (the mapping can be both one-to-one and one-to-many) and adjusting the weights and parameters of the spiking neurons according to the trained analogue network. Pérez-Carrasco first proposed the conversion approach in \cite{perez-carrascoMappingFrameDriven13}. This approach can enhance the performance of hardware implementations of deep neural networks. An advantage of this approach is that the best, state-of-the-art deep networks can be transformed into SNNs without previous modifications and the efficiency of the final SNN is approaching the efficiency of the original network. However not every ANN can be converted, because features or methods which are common in analogue networks might not have its spiking equivalents. \cite{rueckauerConversionContinuousValuedDeep2017} introduced additional techniques to convert more general class of ANNs by implementing features such as soft-max, batch normalisation or max-pooling (using a maximum filter for spatial reduction of the input feature set), which were previously not available in the spiking networks. The resulting accuracy of experiments which used the conversion approach is very promising \cite{tavanaeiDeepLearningSpiking2019, pfeifferDeepLearningSpiking2018}. One of the disadvantages of the conversion is that the most often used technique uses rate codes which are quite inefficient. Rate-coding means that the activation values of the original analogue deep network translate into firing rates of the spiking neurons, so multiple spikes are necessary to stand for a single activation value. Alternative spike codes need to be developed to overcome this. \cite{zambranoFastEfficientAsynchronous2016} proposes one such coding, which maintains the accuracy of the modelled artificial network while reducing the total count of spikes needed in comparison to other conversion methods. \par
There exists a modified hybrid approach known as \textit{constrain-then-train}. Instead of starting with conventional learning of the ANN, additional constraints needed for spiking networks simulation (or neuromorphic hardware) are applied on the first network before its training. Then, the learned parameters of such constrained ANN can be used directly during the mapping onto SNN without further re-scaling \cite{pfeifferDeepLearningSpiking2018}. The retrieved parameters are particular for just single settings of the spiking neuron model, so the network needs retraining if these parameters should change. The final network has an advantage that it adapts better to the target environment than generally converted networks. \par
Above, four approaches to supervised learning of the spiking neural networks are briefly described. The spike-based learning and local learning methods have a common need for improvement of the accuracy to match the results retrieved by the other two methods. However, they offer other notable features such as potentially better energy efficiency and the ability to exploit additional features of the neuromorphic hardware (e.g. precise timing). Also, inferior performance efficiency is being improved by recent works. On the other hand, the conversion techniques are comparable to its original artificial network models, but due to the used coding, its energy efficiency deteriorates.

\subsection{Binary deep neural networks}
Binary deep neural networks (BNNs) are an alternative to spiking neural networks in terms of energy efficiency. BNNs are deep neural networks that use binary activation values and binary-valued weights \cite{simonsReviewBinarized19}. The state-of-the-art binary neural networks achieve slightly degraded accuracy in comparison with non-binary full precision networks, but with reduced memory requirements and less power consumption \cite{courbariauxBinarizedNeural16}. This advantage is more evident if both activation values and weights are binary as more complicated arithmetic operations can be replaced with simpler bitwise operations \cite{kimBitwiseNeural16}. With binary activations, BNNs can also directly use spikes from event-based hardware \cite{linAccurateBinary17}. Compared to spiking networks, the information is still propagated synchronously and does not allow the fast propagation of most salient features \cite{pfeifferDeepLearningSpiking2018}.

\subsection{Comparison}
Standard classification benchmarks are usually used to compare the individual works, although \cite{pfeifferDeepLearningSpiking2018} points out that these benchmarks might not be appropriate for evaluation of spiking networks and should be taken as a proof of concept. The most used benchmark for demonstrating the performance of the spiking network is the MNIST data set of handwritten digits \cite{lecunGradientbasedLearning98}. \Cref{tab:MNIST_benchmark} compares some of the reported classification accuracies on the MNIST data set.

\begin{table}[htbp]
    \centering
    \begin{tabularx}{\linewidth}{>{\raggedright\arraybackslash}Xcc}
    \toprule
        description & type & accuracy (\%) \\
        \midrule
        Rate-based conversion of 7-layer CNN with max-pooling \cite{rueckauerConversionContinuousValuedDeep2017} & SNN & 99.44 \\
        Converted SNN with pulsed Sigma-Delta coding scheme \cite{zambranoFastEfficientAsynchronous2016} & SNN & 99.14 \\
        Converted Lenet-5 with TTFS temporal coding \cite{rueckauerConversionAnalogSpiking2018} & SNN & 98.57 \\
        Spiking CNN with low-pass backpropagation rule \cite{leeTrainingDeepSpiking2016} & SNN & 99.31 \\
        Branching \& Merging CNN with Homogenous filter capsules \cite{kalganovaBranchingMerging20} & ANN & 99.79 \\
        Multi-column deep neural network \cite{cireganMulticolumnDeep12} & ANN & 99.77 \\
        \bottomrule
    \end{tabularx}
    \caption{Comparison of selected deep neural networks.}
    \label{tab:MNIST_benchmark}
\end{table}
\chapter{Comparison of available simulation platforms} \label{chapter:03}

Currently, there exist numerous platforms for simulation of biological neural networks. The first objective is to analyse those platforms from the perspective of capabilities and choose the most suitable of them. The main criterion is the current state of the community and user base around the platform, their ability to provide support to the users and state of the project documentation. Because many of the projects are using GitHub to host its repositories, the metrics like code frequency, contributors count, number of issues and similar indicators displayed by GitHub can help to determine this. The next criterion is whether the platform provides means to interact with both spiking neural networks and artificial neural networks. Other criteria are user-friendliness of the platform and license conditions. The Neural Simulation Tool (NEST) \cite{jordanNEST182019}, Brian \cite{stimbergBrianIntuitiveEfficient2019}, Neuron \cite{carnevaleNEURONBook06} and Nengo \cite{bekolayNengoPythonTool2014} are platforms, which were analysed more precisely. Multiscale Object-Oriented Simulation Environment (MOOSE), STochastic Engine of Pathway Simulation (STEPS), Topographica, Neocortical Simulator (NCS), The Parallel Circuit SIMulator (PCSIM) and the GEneral NEural SImulation System (GENESIS) were other considered simulators. However, they were not included out of consideration for their small community, range of supported models, or because they are no longer actively developed.

\section{The Neural Simulation Tool (NEST)}

NEST is a simulator for spiking neural network models. NEST provides approximately 50 neuron models and over ten synapse models. Its primary focus is on the dynamics, size and structure of the network rather than on the exact morphology of individual neurons. NEST is implemented in C++ and contains a scripting interface for Python and its native simulation language interpreter (SLI). Recently, the NEST initiative which coordinates the development if NEST developed a new markup language NESTML to make the creation of new neuron models easier \cite{plotnikovNESTMLModelingLanguage2016}.  NEST can be used on many computer architectures from low-end laptops to supercomputers. Project's page explicitly mentions Linux, MacOS X and IBM BlueGene as supported platforms. Windows operating system can utilise a prepared virtual machine image. The scaling of the existing model on a high-performance cluster or multi-core computer is convenient and minimal, or no changes are required, which is an advantage over other simulation platforms \cite{tikidji-hamburyanSoftwareBrainNetwork2017}. NEST is applied in projects like the Human Brain Project or BrainScaleS. NEST appears in over 290 papers as the simulator which was used \cite{julichaachenresearchallianceNESTNeuralSimulation2015}.

\section{Brian}

Brian is a spiking neural network simulator written in Python. It is an equation oriented simulator, i.e. neural models are defined in its mathematical form with differential equations rather than written in programming language function. This feature makes the simulator very flexible. To increase the speed of simulation, Brian can run in several runtime code generation modes, for example, NumPy, Cython or generate a standalone C++ code (although some limitations apply). A Downside of the Brian software is limited support for parallel computations and total lack of support for cluster computations \cite{tikidji-hamburyanSoftwareBrainNetwork2017}. Project's page does not state any numbers of its users, but there exists a Google group \cite{Brian} with support topics, which has over 300 users. Maintainers of the project do not track any information about publications, which use Brian.

\section{NEURON}

NEURON is a simulation environment for modelling single neurons or neural networks. It specialises in empirically-based simulations. NEURON is more suitable for simulating single neuron models with more complex structure. NEURON contains a graphical user interface (GUI), which can be used to manipulate an existing model or create a new one. The GUI can be used to create both single neuron model and a network model. Neural networks and neuron models can be described with NEURON's own interpreted language called "hoc" or with Python script. New modules for neuron dynamics can be developed in its native simplified language NMODL \cite{tikidji-hamburyanSoftwareBrainNetwork2017}. NEURON utilises several methods to optimise the efficiency of its code. For example, it provides routines for tabulating values of often used equations to avoid their computation every time. It also supports optimisations for high-performance computing, such as embarrassingly parallel computation without the need to use other software. NEURON is the only simulator which can distribute computations for a single complex neuron over a cluster \cite{tikidji-hamburyanSoftwareBrainNetwork2017}. According to its web page, its user base is extensive with more than 1500 registered users at the forum and more than 1900 of scientific articles that reported the use of NEURON. Software installers are available for MS Windows, macOS and Linux.

\section{Nengo}

Nengo is a tool for simulating large neural models. Nengo exploits the theoretical modelling approach described by the Neural Engineering Framework. The Neural Engineering Framework proposes three principles (representation, transformation, and dynamics) for the construction of large-scale neural models which are simulator independent. Nengo decouples model creation and simulation so that multiple simulators can run created models. NengoDL is one such simulator built on top of the deep learning framework TensorFlow. Nengo also provides a backend for neuromorphic hardware Intel Loihi. From version 2 onwards, Nengo is written in the Python programming language. It provides packages for all major operating systems. Figure \ref{fig:nengo_ecosystem} shows a relation between the Nengo components. Nengo incorporates a proprietary license which allows free manipulation with Nengo and its source code for any non-commercial purpose as long as the copyright notice is included. According to the project's webpage, more than 100 publications mentioned Nengo. Nengo forum has over 200 registered users of which about 40 were active last month.

\begin{figure}[ht]
    \centering
    \includesvg[width=\textwidth]{nengo_ecosystem}
    \caption{The Nengo ecosystem. Source: \url{https://www.nengo.ai/documentation/}}
    \label{fig:nengo_ecosystem}
\end{figure}

\subsection{Neural engineering framework}
The neural engineering framework (NEF) is a general method for building neural models. It offers a method to solve connection weights between network components by specifying properties of the used neuron models, the values which they represent and the functions which should be computed \cite{stewartTechnicalOverviewNeural2012}. The NEF defines three principles for building large-scale neural models. These principles are called representation, transformation and dynamics. \par
The representation principle states that a population of neurons represents information. Neural populations represent time-varying signals through their spiking responses, and a signal is regarded as a vector of real numbers of arbitrary length.  This way, neural computations can be defined by manipulating the information using conventional mathematics \cite{bekolayNengoPythonTool2014}. During the encoding process, a specific amount of current is released into a single neuron. The tuning curve of the neuron determines how likely it is that the neuron spike as a function of the input signal. The spike frequency of the population represents the encoded information. The information can be decoded by observing these spike trains. First, the spike frequencies are filtered with exponentially decaying filter. The decoding weights multiply the filtered spike trains and then the spike trains sum together. The accuracy of the decoded information increases with the increasing number of neurons in the population. \par
The transformation principle describes how to decode spike trains to compute transformations of signals. It provides a way to compute the connection weights between populations to represent an arbitrary function. \par
When the neurons are connected recurrently, the vectors represented by neural populations can be interpreted as state variables of the dynamical system and methods of control theory can be applied \cite{bekolayNengoPythonTool2014}.

\subsection{Semantic pointer architecture}
The semantic pointer architecture provides an approach to build cognitive models with large-scale spiking neural networks. As stated in \cite{eliasmithHowBuild13}, the semantic pointers are neural representations that carry partial semantic content and are composable into the representational structures to support complex cognition. The Semantic Pointer Architecture (SPA) uses vectors to represent concepts. The Nengo-SPA package contains the implementation of the SPA for the Nengo simulator. \par
Nengo used with SPA is an underlying foundation for the proof-of-concept Spaun 2.0 \cite{chooSpaunExtending18} model, which is according to the author, the largest functional brain model. The Spaun 2.0 model consists of several neural networks, each specialised for a different cognitive task. Spaun 2.0 deals with digit recognition, list memory, question answering and other tasks.

\section{Summary of platform comparison}
Out of available neural simulation platforms, four were chosen as the best candidates for further analysis. As a starting point for the comparison of Neuron, Brian and NEST, a previous comparative study \cite{tikidji-hamburyanSoftwareBrainNetwork2017} was used. Each of the platforms has some sort of forum or mailing list for communication of its users. Statistics retrieved from these pages were used to compare the number of users of the platform. Other statistics were retrieved from the GitHub sites of the projects. They can be used as an indicator of the activity of the projects in the last years. The collected statistics are shown in appendix \ref{appendix:b} below. The Neuron simulator has by far the largest community on its user forum and is cited by other researchers most often. It is probably because the Neuron simulator is developed for the longest time. \par
On the other hand, The NEST simulator has almost three times more contributors to its source code than any other platform. Nengo is the only simulator which offers utilisation of the artificial neural networks to improve simulations of the spiking neural networks. A table in appendix \ref{appendix:c} contains information about the state of the documentation of the four compared projects. Appendix \ref{appendix:d} contains a table with an overview of some additional characteristics of the simulators.
\chapter{Other libraries} \label{chap:04}

This chapter contains a short description of the rest of the software, libraries and frameworks encountered during research. PyNN \cite{davisonPyNNCommonInterface2009}, a simulator-independent language for building neuronal network models, is mentioned at the end of this chapter. PyNN works as a frontend for Neuron, NEST and Brian simulators mentioned in the previous chapter.

\section{SNN toolbox}
SNN toolbox (SNN-TB) \cite{rueckauerConversionContinuousValuedDeep2017} is a conversion tool which automates the process of conversion of trained ANN to an SNN. The toolbox uses a deep learning model written in one of the supported ANN frameworks (Currently Keras~/~TensorFlow \cite{cholletKeras15}, Lasagne \cite{sanderdielemanLasagneFirst15}, Caffe \cite{jiaCaffeConvolutional14} and PyTorch \cite{paszkePyTorchImperative19}). The provided input model is parsed, and a general abstract model in Keras is derived. This internal model is used to achieve the actual conversion to the spiking network. That is achieved by replacing the analogue neurons to spiking integrate-and-fire neurons and altering the connection weights correspondingly. The resulting spiking model can be exported to several spiking simulators or deployed on dedicated neuromorphic chip-sets. Currently, the supported simulation platforms are PyNN, Brian2 and built-in MegaSim and INIsim simulators. Intel Loihi and SpiNNaker project are the supported hardware platforms at the moment. \Cref{fig:snn-tb_workflow} illustrates the workflow of the SNN toolbox.

\begin{figure}[htbp]
    \centering
    \includegraphics[width=\textwidth]{SNN-TB_workflow}
    \caption{The illustration of the SNN toolbox workflow. An arbitrary supported deep learning model is transformed into an abstract Keras model, which is subsequently converted to a spiking network.
    Source: \url{https://snntoolbox.readthedocs.io/en/latest/guide/intro.html}}
    \label{fig:snn-tb_workflow}
\end{figure}

\section{Keras / TensorFlow}
Keras is a high-level deep learning API which can be used with multiple neural network or machine learning toolkits, mainly with TensorFlow. TensorFlow is an open-source machine learning platform. It consists of an interface for formulating machine learning algorithms and system-specific implementations to perform such algorithms.

\section{PyNN}
PyNN is a language for building models of neuronal networks independently of the used simulator. It provides a common API in the Python language to write the code and run it on multiple backends. PyNN API is designed to support neural networks models at a high-level of abstraction. Currently, Neuron, Brian (version 1) and NEST are supported simulators. PyNN also supports the SpiNNaker and BrainScaleS neuromorphic hardware and is partially compatible with NeuroML model description language. Figure in \cref{fig:pynn} shows the architecture of the PyNN interface \cite{bruderleNeuroscientificModelingMixedSignal2009}.

\begin{figure}[htb]
    \centering
    \begin{sideways}
        \includesvg[inkscapelatex=false,width=0.72\textheight]{PyNN_architecture}
    \end{sideways}
    \caption{PyNN architecture}
    \label{fig:pynn}
\end{figure}

\chapter{Description of exploratory use cases} \label{chap:05}

This chapter describes the initial experiments with the Nengo simulator and the SNN-Toolbox, which were selected as the most appropriate tools for subsequent use in this work. The decision was based on the pieces of information stated about the individual tools in the previous chapters. Both tools make it possible to use analogue neural networks to train the model and then convert it to the corresponding spiking neural network for inference. This approach is more utilisable for practical needs at the moment because other approaches of using SNNs suffer from issues described in \cref{chap:02} above. Thus, it is complicated to use them for any practical application such as image recognition. However, they are still necessary for research in the area of spiking networks and scientific experiments. Furthermore, the other simulators proved to be quite hard to work with, for example because of unreliable interfaces of their dependencies or other issues. \par
In the beginning, the Nengo simulator was inspected more thoroughly. It helps to new users with its simple graphical interface, which contains several tutorials and examples. The tutorials focus mainly on an explanation of the Neural Engineering Framework and Semantic Pointer Architecture methods introduced by authors of the Nengo simulator, rather than on SNNs generally. The GUI is divided into a script editor and visualisation area, where are visualised the objects created in the script editor in actual time. These visualisations help understand the whole course of the simulation of the spiking neural network. An additional package \texttt{nengo-gui} must be installed to access the graphical environment. The environment is an HTML5-based application, which is accessed through an internet browser. The interactive editor supports Nengo and NengoDL backends for running the simulation. \par
% TODO describe Spaun execution details
The next goal was to execute the Spaun 2.0 model to inspect the capabilities of the Nengo even further. The objective was to simulate the handwritten digit recognition, which is one of the cognitive tasks that Spaun 2.0 supports. Although this use case should be quite straightforward according to the documentation of the Nengo, there appeared a few issues. Firstly, there does not exist any documentation for the model. It means that the only way how to get information about its execution is to correspond with its authors about every detail, for example, to discover that the model is not compatible with recent versions of the Nengo package. The other problem is probably caused by the fact that the Spaun 2.0 model can require more than 26 gigabytes of memory to execute. Even though the model was executed on a cloud platform, which provides such resources, the program still finished the simulation with memory allocation errors.

\section{Using the same pre-trained model for the Nengo and SNN-Toolbox}
The next objective was to assess conversion abilities of both Nengo and SNN conversion toolbox on a similar use case. Conversion of a simple neural network model represented by Keras interface was selected as a suitable use case. The model was assembled and trained as an ordinary analogue neural network on an MNIST dataset. The model architecture must have been constrained to allow comparison of the two convertors because Nengo currently does not support batch normalization. Although it should fall back to ANN implementation for unsupported layers, it did not work if batch normalization layers were in the model.\par
There was an issue with using the same network model directly because there exist two versions of Keras framework at this time. SNN-TB uses a standalone multi-backend version of Keras, which supports Theano and CNTK toolkits alongside with the TensorFlow backend. However, this version of Keras will not be further developed in the future. The other version of Keras is integrated directly into TensorFlow as its high-level API for neural network development. This version of Keras interface is used in Nengo platform. However, both versions work similarly, and the model exported from the multi-backend version can be migrated into the other without further complications.

\subsection{Model architecture}
The final selected model was a convolutional network, which consisted of multiple consecutive convolutional layers, and contained an average pooling layer. A visualisation of the network architecture is shown in \cref{fig:keras_model}. The first convolution layer was consisted of thirty-two filters. The kernel size was $3 \times 3$. The second convolutional layer had sixty-four $3 \times 3$ filters with a stride length of $2 \times 2$. All layers have rectified linear unit (ReLU) as activation function, and the weight matrices were initialized with He uniform initializer. The only exception is the last dense layer, which uses softmax activation. The presented model was configured to use the Adam \cite{kingmaAdamMethod17} optimizer and categorical cross-entropy as a loss function. The trained Keras model was exported to the filesystem. \par
\begin{figure}[htbp]
    \centering
    \includesvg[inkscapelatex=false, height=0.65\textheight]{images/keras_model}
    \caption{The Keras model of a convolutional network used for classification of the MNIST dataset.}
    \label{fig:keras_model}
\end{figure}
Conversion of the network with SNN-TB is quite straightforward. There must be created a configuration file, which contains all necessary parameters that must be passed to the \texttt{main} function of the toolbox. The configuration file can be written with a few simple utility functions of the SNN-TB. The configuration file contains paths to the model and dataset as well as a specification of the used simulator and its parameters, parameters of the simulated spiking neuron cells or plots and variable logs, which should be generated by the toolbox. The SNN-TB then automatically performs the conversion of the network and evaluates it on the given test samples. There is an option to evaluate the ANN input model so the results can be compared instantly. \par
% TODO describe precise steps to convert and simulate SNN with Nengo
Some adjustments need to be done to simulate a corresponding spiking network in the Nengo simulator. The input samples must be reformatted to conform with the format which is required by the Nengo simulator. More precisely, there must be added a dimension to the data array to spread the data over time of the simulation, and the input image needs to be flattened into a single dimension. The \texttt{nengo\_dl.Converter} class converts the ANN. Then the simulation is executed the same way as other simulations in Nengo.

\chapter{Conversion of the convolutional neural network for classification on large multi-subject P300 dataset} \label{chap:06}

A convolutional neural network described in \cite{varekaEvaluationConvolutional20} was converted with the SNN conversion toolbox. SNN-TB was selected for the conversion over Nengo simulation platform because the CNN contained batch normalization layer and softmax activation function, which prevented Nengo to convert the network successfully. \par
The original CNN was used for binary classification of event-related potential (ERP) signals retrieved from large multi-subject P300 dataset, which was presented in \cite{moucekEventrelatedPotential17}. The dataset was retrieved from EEG experiments, where subjects were told to concentrate on a single-digit number (from 0 to 9). Experimenters were trying to guess the number, which the subject selected, from recorded EEG data while the subject was stimulated with random sequences of all digits. \cite{varekaEvaluationConvolutional20} extracted short intervals of signal around all target stimuli from this dataset. An equal amount of non-target samples was acquired randomly from the remaining data. This newly-emerged dataset consisted of two distinct equal-sized classes, the target epochs (i.e. short time intervals of EEG data around the stimulus, which was the subject's thought number) and the non-target epochs. Each epoch contained three EEG channels and was 1200 ms long. The sampling frequency was 1 kHz. That means that each sample of the network input was a $3 \times 1200$ matrix. \par
The input data were preprocessed in a similar fashion as in \cite{varekaEvaluationConvolutional20}. For the purpose of this work was used the already modified subset \cite{moucekReplicationData19} of the original dataset, which was created during the classification with the original CNN and was publicly available. The input file contained one matrix of feature vectors for all target data and one matrix of feature vectors for all non-target data. Those matrices were concatenated to create the input vectors. The corresponding ground truth labels were produced by one-hot-encoding of the input data. Then, severely damaged epochs were discarded from the input the same way as in the original article \cite{varekaEvaluationConvolutional20}. It was done by discarding all epochs, where amplitudes were higher than 100 $\mu V$ in any of the three channels. These preprocessed data were divided into training and test subsets by randomly picking out 25\% of the dataset samples into the test set. The test set was saved to the filesystem so it could be passed to SNN-TB during conversion later. \par
The network architecture of the CNN model was replicated entirely from the original work. This model was cross-validated the same way as in the original article. That means 30 iterations of the Monte-Carlo cross-validation were performed, and each iteration held out 25\% of the training set to create validation subset. In each iteration, a new instance of the network was created and trained. The training was done in 30 epochs while early stopping with patience value of 5 was used. After that, the trained ANN was evaluated on the test set. 10\% of the dataset samples, which were used during the training, were saved for normalization of the layer weights. SNN-TB did this normalization before the conversion process. The conversion toolbox parsed the evaluated ANN into its internal Keras representation, weights of the parsed model were normalized with previously saved part of the training data. The normalized model was converted into a spiking model, and the spiking model was evaluated on the test set each iteration. The evaluation of each input sample of the spiking network was simulated for 50 timesteps. The simulator evaluated the whole test set gradually in batches of 49 samples. \par
Accuracies from the 30 iterations of cross-validation were recorded for both networks. The other metrics, which are showed in the article about the original neural network (AUC, precision, recall), could not be computed, because SNN-TB does not provide a simple way to define arbitrary metrics at this time \footnotemark.

\footnotetext{This issue was discussed with a contributor to the SNN conversion toolbox on the project's GitHub page. The discussion did not lead to a successful implementation of the custom metrics before this work was completed. A link to the website \url{https://github.com/NeuromorphicProcessorProject/snn_toolbox/issues/56} is provided with a hope that the solution might appear there in future.}

\section{Results}
The average accuracy and sample standard deviation from the 30 cross-validation iterations of the experiment are shown in \cref{tab:GTN_results}. \Cref{fig:GTN_graph} shows a comparison of the accuracies from the individual iterations. The decrease of accuracy on the SNN was expected as previous works from the field reports similar drops of performance. Also, this experiment validated only a single combination of network parameters, which is stated in the previous section. There are plenty of parameters (for the used neuron type, network or the whole simulator), so there surely is a space for fine-tuning and optimization. Whole other area of experimentation would appear if the SNN-TB was used with any other supported backend. That would possibly allowed using other neuron types.

\begin{table}[htbp]
    \centering
    \begin{tabularx}{\linewidth}{>{\raggedright\arraybackslash}Xcc}
    \toprule
        Model & Average test accuracy & Sample standard deviation \\
    \midrule
        Original model & 0.6370 & 0.0066 \\
        Converted model & 0.5720 & 0.0156 \\
    \bottomrule
    \end{tabularx}
    \caption{Average accuracy and sample standard deviation of the classification results received during cross-validation.}
    \label{tab:GTN_results}
\end{table}

\begin{figure}[htbp]
    \centering
    \includesvg[width=\linewidth]{GTN_graph}
    \caption{The accuracies across all iterations of the experiment.}
    \label{fig:GTN_graph}
\end{figure}


\chapter{Conclusion} \label{chap:07}

This thesis clarifies the main distinctions between analogue neural networks and spiking neural networks, depict the current state of research in the area of the spiking networks and demonstrate the findings on a selected machine learning problem with the help of selected instruments. It was described how spiking networks can be beneficial in research for their biological similarities with biological neural systems, and their fitness for utilization in embedded applications. The work later focuses on the collation of the most advanced simulators of spiking networks. Nengo ecosystem and SNN-Toolbox were selected for the subsequent experiments. It was outlined how these tools work on a use case where a base analogue network was converted with both tools and evaluated on the MNIST image classification dataset. It was shown that both instruments were able to convert an original non-spiking network to a spiking alternative with some modifications. There were also described personal experiences and impressions from using the tools. Results of both converted networks were comparable with results of the base model. \par
The main contribution of this work is a conversion of an earlier convolutional network for event-related potentials classification. The achieved average accuracy (57.2\%) do not surpass the accuracy of the original network. Values of other metrics could not be compared with original because the used conversion toolbox does not make it possible to calculate those metrics. The results of this experiment could be improved with a more profound research of the effects of individual simulation parameters on the performance of the spiking networks and using algorithms for optimization of those parameters. Another improvement in the classification of event-related potentials could be achieved if a new model, which would exploit specific qualities of the spiking networks, was created. \par
This work focused on computer simulations of the spiking networks and overlooked available neuromorphic platforms such as Intel Loihi or SpiNNaker chips. It would be an impressive continuation of this work to experiment with these novel hardware architectures. \par
All source codes of the used and created applications were made public \footnotemark. Even though, the achieved results of any of the applications are not ground breaking, the work and the personal experiences with the mentioned tools might be helpful to subsequent works in this field.
\footnotetext{All materials related to this thesis were published on \url{https://github.com/RKCZ/Extension-of-neural-network-architecture}.}

% 
% PRO ANGLICKOU SAZBU JE NUTNÉ ZMĚNIT
% CITAČNÍ STYL!
%
\emergencystretch=1em

\printbibliography[
heading=bibintoc
]

% \bibliographystyle{ACM-Reference-Format}
% {\raggedright\small
% \bibliography{literature}
% }

\end{document}
