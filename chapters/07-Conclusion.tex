\chapter{Conclusion} \label{chap:07}

This thesis clarifies the main distinctions between analogue neural networks and spiking neural networks, depict the current state of research in the area of the spiking networks and demonstrate the findings on a selected machine learning problem with the help of selected instruments. It was described how spiking networks could be beneficial in research for their biological similarities with biological neural systems, and their fitness for utilisation in embedded applications. The work later focuses on the collation of the most advanced simulators of spiking networks. Nengo ecosystem and SNN-Toolbox were selected for the subsequent experiments. It was outlined how these tools work on a use case where a base analogue network was converted with both tools and evaluated on the MNIST image classification dataset. It was shown that both instruments were able to convert an original non-spiking network to a spiking alternative with some modifications. There were also described personal experiences and impressions from using the tools. The results of both converted networks were comparable with the results of the base model. \par
The main contribution of this work is a conversion of an earlier convolutional network for event-related potentials classification. The achieved average accuracy (57.2\%) does not surpass the accuracy of the original network. Values of other metrics could not be compared with original work because the used conversion toolbox does not make it possible to calculate those metrics. The results of this experiment could be improved with a more profound research of the effects of individual simulation parameters on the performance of the spiking networks and using algorithms for optimization of those parameters. Another improvement in the classification of event-related potentials could be achieved if a new model, which would exploit specific qualities of the spiking networks, was created. \par
This work focused on computer simulations of the spiking networks and overlooked available neuromorphic platforms such as Intel Loihi or SpiNNaker chips. It would be an impressive continuation of this work to experiment with these novel hardware architectures. \par
All source codes of the used and created applications were made public\footnotemark. Even though the achieved results of any of the applications are not groundbreaking, the work and the personal experiences with the mentioned tools might be helpful to subsequent works in this field.
\footnotetext{All materials related to this thesis were published on \url{https://github.com/RKCZ/Extension-of-neural-network-architecture}.}