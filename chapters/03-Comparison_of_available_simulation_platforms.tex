\chapter{Comparison of available simulation platforms} \label{chapter:03}

Currently, there exist numerous platforms for simulation of biological neural networks. The first objective is to analyse those platforms from the perspective of capabilities and choose the most suitable of them. The main criterion is the current state of the community and user base around the platform, their ability to provide support to the users and state of the project documentation. Because many of the projects are using GitHub to host its repositories, the metrics like code frequency, contributors count, number of issues and similar indicators displayed by GitHub can help to determine this. The next criterion is whether the platform provides means to interact with both spiking neural networks and artificial neural networks. Other criteria are user-friendliness of the platform and license conditions. The Neural Simulation Tool (NEST) \cite{jordanNEST182019}, Brian \cite{stimbergBrianIntuitiveEfficient2019}, Neuron \cite{carnevaleNEURONBook06} and Nengo \cite{bekolayNengoPythonTool2014} are platforms, which were analysed more precisely. Multiscale Object-Oriented Simulation Environment (MOOSE), STochastic Engine of Pathway Simulation (STEPS), Topographica, Neocortical Simulator (NCS), The Parallel Circuit SIMulator (PCSIM) and the GEneral NEural SImulation System (GENESIS) were other considered simulators. However, they were not included out of consideration for their small community, range of supported models, or because they are no longer actively developed.

\section{The Neural Simulation Tool (NEST)}

NEST is a simulator for spiking neural network models. NEST provides approximately 50 neuron models and over ten synapse models. Its primary focus is on the dynamics, size and structure of the network rather than on the exact morphology of individual neurons. NEST is implemented in C++ and contains a scripting interface for Python and its native simulation language interpreter (SLI). Recently, the NEST initiative which coordinates the development if NEST developed a new markup language NESTML to make the creation of new neuron models easier \cite{plotnikovNESTMLModelingLanguage2016}.  NEST can be used on many computer architectures from low-end laptops to supercomputers. Project's page explicitly mentions Linux, MacOS X and IBM BlueGene as supported platforms. Windows operating system can utilise a prepared virtual machine image. The scaling of the existing model on a high-performance cluster or multi-core computer is convenient and minimal, or no changes are required, which is an advantage over other simulation platforms \cite{tikidji-hamburyanSoftwareBrainNetwork2017}. NEST is applied in projects like the Human Brain Project or BrainScaleS. NEST appears in over 290 papers as the simulator which was used \cite{julichaachenresearchallianceNESTNeuralSimulation2015}.

\section{Brian}

Brian is a spiking neural network simulator written in Python. It is an equation oriented simulator, i.e. neural models are defined in its mathematical form with differential equations rather than written in programming language function. This feature makes the simulator very flexible. To increase the speed of simulation, Brian can run in several runtime code generation modes, for example, NumPy, Cython or generate a standalone C++ code (although some limitations apply). A Downside of the Brian software is limited support for parallel computations and total lack of support for cluster computations \cite{tikidji-hamburyanSoftwareBrainNetwork2017}. Project's page does not state any numbers of its users, but there exists a Google group \cite{Brian} with support topics, which has over 300 users. Maintainers of the project do not track any information about publications, which use Brian.

\section{NEURON}

NEURON is a simulation environment for modelling single neurons or neural networks. It specialises in empirically-based simulations. NEURON is more suitable for simulating single neuron models with more complex structure. NEURON contains a graphical user interface (GUI), which can be used to manipulate an existing model or create a new one. The GUI can be used to create both single neuron model and a network model. Neural networks and neuron models can be described with NEURON's own interpreted language called "hoc" or with Python script. New modules for neuron dynamics can be developed in its native simplified language NMODL \cite{tikidji-hamburyanSoftwareBrainNetwork2017}. NEURON utilises several methods to optimise the efficiency of its code. For example, it provides routines for tabulating values of often used equations to avoid their computation every time. It also supports optimisations for high-performance computing, such as embarrassingly parallel computation without the need to use other software. NEURON is the only simulator which can distribute computations for a single complex neuron over a cluster \cite{tikidji-hamburyanSoftwareBrainNetwork2017}. According to its web page, its user base is extensive with more than 1500 registered users at the forum and more than 1900 of scientific articles that reported the use of NEURON. Software installers are available for MS Windows, macOS and Linux.

\section{Nengo}

Nengo is a tool for simulating large neural models. Nengo exploits the theoretical modelling approach described by the Neural Engineering Framework. The Neural Engineering Framework proposes three principles (representation, transformation, and dynamics) for the construction of large-scale neural models which are simulator independent. Nengo decouples model creation and simulation so that multiple simulators can run created models. NengoDL is one such simulator built on top of the deep learning framework TensorFlow. Nengo also provides a backend for neuromorphic hardware Intel Loihi. From version 2 onwards, Nengo is written in the Python programming language. It provides packages for all major operating systems. Figure \ref{fig:nengo_ecosystem} shows a relation between the Nengo components. Nengo incorporates a proprietary license which allows free manipulation with Nengo and its source code for any non-commercial purpose as long as the copyright notice is included. According to the project's webpage, more than 100 publications mentioned Nengo. Nengo forum has over 200 registered users of which about 40 were active last month.

\begin{figure}[ht]
    \centering
    \includesvg[width=\textwidth]{nengo_ecosystem}
    \caption{The Nengo ecosystem. Source: \url{https://www.nengo.ai/documentation/}}
    \label{fig:nengo_ecosystem}
\end{figure}

\subsection{Neural engineering framework}
The neural engineering framework (NEF) is a general method for building neural models. It offers a method to solve connection weights between network components by specifying properties of the used neuron models, the values which they represent and the functions which should be computed \cite{stewartTechnicalOverviewNeural2012}. The NEF defines three principles for building large-scale neural models. These principles are called representation, transformation and dynamics. \par
The representation principle states that a population of neurons represents information. Neural populations represent time-varying signals through their spiking responses, and a signal is regarded as a vector of real numbers of arbitrary length.  This way, neural computations can be defined by manipulating the information using conventional mathematics \cite{bekolayNengoPythonTool2014}. During the encoding process, a specific amount of current is released into a single neuron. The tuning curve of the neuron determines how likely it is that the neuron spike as a function of the input signal. The spike frequency of the population represents the encoded information. The information can be decoded by observing these spike trains. First, the spike frequencies are filtered with exponentially decaying filter. The decoding weights multiply the filtered spike trains and then the spike trains sum together. The accuracy of the decoded information increases with the increasing number of neurons in the population. \par
The transformation principle describes how to decode spike trains to compute transformations of signals. It provides a way to compute the connection weights between populations to represent an arbitrary function. \par
When the neurons are connected recurrently, the vectors represented by neural populations can be interpreted as state variables of the dynamical system and methods of control theory can be applied \cite{bekolayNengoPythonTool2014}.

\subsection{Semantic pointer architecture}
The semantic pointer architecture provides an approach to build cognitive models with large-scale spiking neural networks. As stated in \cite{eliasmithHowBuild13}, the semantic pointers are neural representations that carry partial semantic content and are composable into the representational structures to support complex cognition. The Semantic Pointer Architecture (SPA) uses vectors to represent concepts. The Nengo-SPA package contains the implementation of the SPA for the Nengo simulator. \par
Nengo used with SPA is an underlying foundation for the proof-of-concept Spaun 2.0 \cite{chooSpaunExtending18} model, which is according to the author, the largest functional brain model. The Spaun 2.0 model consists of several neural networks, each specialised for a different cognitive task. Spaun 2.0 deals with digit recognition, list memory, question answering and other tasks.

\section{Summary of platform comparison}
Out of available neural simulation platforms, four were chosen as the best candidates for further analysis. As a starting point for the comparison of Neuron, Brian and NEST, a previous comparative study \cite{tikidji-hamburyanSoftwareBrainNetwork2017} was used. Each of the platforms has some sort of forum or mailing list for communication of its users. Statistics retrieved from these pages were used to compare the number of users of the platform. Other statistics were retrieved from the GitHub sites of the projects. They can be used as an indicator of the activity of the projects in the last years. The collected statistics are shown in appendix \ref{appendix:b} below. The Neuron simulator has by far the largest community on its user forum and is cited by other researchers most often. It is probably because the Neuron simulator is developed for the longest time. \par
On the other hand, The NEST simulator has almost three times more contributors to its source code than any other platform. Nengo is the only simulator which offers utilisation of the artificial neural networks to improve simulations of the spiking neural networks. A table in appendix \ref{appendix:c} contains information about the state of the documentation of the four compared projects. Appendix \ref{appendix:d} contains a table with an overview of some additional characteristics of the simulators.