\chapter{Description of experimental use cases} \label{chapter:05}

This chapter describes the initial experiments with the Nengo simulator and the SNN-Toolbox, which were selected as the most appropriate tools for subsequent use in this work. The decision was based on the pieces of information stated about the individual tools in the previous chapters. Both tools aim to use analogue neural networks to train the model and then convert it to the spiking alternative for inference. This approach is more utilisable for practical needs at the moment because other approaches to use SNNs suffer from issues described in chapter \ref{chapter:02} above. Thus, it is complicated to use them for any practical application such as image recognition. However, they are still necessary for research in the area of spiking networks and scientific experiments. Furthermore, the other simulators proved to be quite hard to work with due to inconsistent dependencies or other issues. \par
At the beginning, the Nengo simulator was inspected more thoroughly. It helps to new users with a simple graphical interface which contains several tutorials and examples. The GUI is split into editor and visualisation area, where are shown visualisations of the executed script. These visualisations help understand the whole course of the simulation of the spiking neural network. \par
The next goal was to execute the Spaun 2.0 model to inspect the capabilities of the Nengo even further. The objective was to simulate the handwritten digit recognition which is one of the cognitive tasks that Spaun 2.0 supports. Although, this use case should be quite straightforward according to the documentation of the model, there appeared a few issues. Firstly, the model is not compatible with recent versions of the Nengo package. However, this information is not stated in the documentation. The other problem is probably caused by the fact, that the Spaun 2.0 model can require more than 26 gigabytes of memory to execute. Even though, the model was executed on the Google Colaboratory platform, which provides such resources, the program still failed to finish the simulation successfully.