\chapter{Description of exploratory use cases} \label{chapter:05}

This chapter describes the initial experiments with the Nengo simulator and the SNN-Toolbox, which were selected as the most appropriate tools for subsequent use in this work. The decision was based on the pieces of information stated about the individual tools in the previous chapters. Both tools aim to use analogue neural networks to train the model and then convert it to the spiking alternative for inference. This approach is more utilisable for practical needs at the moment because other approaches to use SNNs suffer from issues described in chapter \ref{chapter:02} above. Thus, it is complicated to use them for any practical application such as image recognition. However, they are still necessary for research in the area of spiking networks and scientific experiments. Furthermore, the other simulators proved to be quite hard to work with due to inconsistent dependencies or other issues. \par
At the beginning, the Nengo simulator was inspected more thoroughly. It helps to new users with its simple graphical interface, which contains several tutorials and examples. The tutorials focus mainly on an explanation of the NEF and SPA methods created by authors of the Nengo, rather than on SNNs generally. The GUI is split into a script editor and visualisation area, where the objects created at the edited script is visualised in real time. These visualisations help understand the whole course of the simulation of the spiking neural network. \par
The next goal was to execute the Spaun 2.0 model to inspect the capabilities of the Nengo even further. The objective was to simulate the handwritten digit recognition, which is one of the cognitive tasks that Spaun 2.0 supports. Although this use case should be quite straightforward according to the documentation of the Nengo, there appeared a few issues. Firstly, there does not exist any documentation for the model. It means that the only way how to get information about its execution is to correspond with its authors about every detail, for example, to discover that the model is not compatible with recent versions of the Nengo package. The other problem is probably caused by the fact that the Spaun 2.0 model can require more than 26 gigabytes of memory to execute. Even though the model was executed on a cloud platform, which provides such resources, the program still finished the simulation with memory allocation errors. \par
The next objective was to assess conversion abilities of both Nengo and SNN-Toolbox on a common use case. As a suitable use case was selected conversion of a simple neural network model represented by Keras interface. The model was assembled and trained as an ordinary analogue neural network on an MNIST dataset. The used model was a convolutional network consisted of multiple consecutive convolutional layers, average pooling layer and batch normalization layer. A visualisation of the model is shown in figure \ref{fig:keras_model}. The presented model was configured to use an Adam optimizer and categorical crossentropy as a loss function.
\begin{figure}[htbp]
    \centering
    \includesvg[inkscapelatex=false, height=0.75\textheight]{images/keras_model}
    \caption{The keras model of a convolutional network used for classification of the MNIST dataset.}
    \label{fig:keras_model}
\end{figure}