\chapter{Introduction} \label{chap:01}

A neural network is a set or population of specialised cells (neurons) that are interconnected by synapses. In biology, the neural network forms the structure of a neural system in animals. The interconnection pattern, size and spatial organisation of the population determines the architecture and specialisation of the network. The network specialises in carrying out a specific function when it is activated. These specialised networks then tie to one another to develop larger systems (e.g. animal brain). These natural phenomena inspired computer scientists to design a set of algorithms that models some aspects of the animal brain. 

Artificial (analogue) neural networks (ANNs) were initially inspired by biological neural systems, but many concepts were simplified or modified to conform to their practical applications. On the other hand, the spiking neural networks (SNNs) were mostly meant to simulate their biological models as closely as possible and thus improve our understanding of how real biological networks achieve their cognitive abilities with incredibly low energy consumption. Although the differences between those approaches still last, some aspects of the two approaches are getting more interconnected as the research advances and one approach can exploit the other. Nowadays, deep ANNs achieve satisfying accuracy in many complex tasks, although with extensive requirements on computation power. Because of that, new goals for the improvement of their efficiency are being set. That is where deep SNNs appear to be an exciting topic of research because of better power efficiency \cite{caoSpikingDeepConvolutional2015, tavanaeiDeepLearningSpiking2019}, although the applications generally do not accomplish the same accuracy yet. Apart from the applications in which ANNs are used, spiking networks open up new fields of utilisation. In \cref{chap:02} is a summary of the principal differences between the two classes of neural networks and a survey of the current state of the research in the area of the spiking neural networks. \Cref{chap:03} compares existing simulators of the spiking neural networks. In \cref{chap:04} are described the remaining tools used in this work. \Cref{chap:05} describes early attempts to simulate spiking networks with selected software.
